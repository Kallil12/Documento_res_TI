% ----------------------------------------------------------
% Introdução (exemplo de capítulo sem numeração, mas presente no Sumário)
% ----------------------------------------------------------
\chapter*{Introdução}
\addcontentsline{toc}{chapter}{Introdução}  
% ----------------------------------------------------------

A Tecnologia da Informação (TI) vem se tornando cada vez mais importante em empresas e órgãos, as aplicações vão desde a infraestrutura que busca conectar os diferentes setores, mantendo a segurança da rede, até a automação de processos. Além disso, nesse espectro de aplicações da TI podemos incluir o melhoramento da gestão usando a computação, atualmente a quantidade de dados e variáveis disponíveis para o gestores é muito grande, e é extremamente difícil de se gerenciar essa massa de dados, para isso existem várias ferramentas que têm por objetivo auxiliar na visualização e futura tomada de decisão dos administradores.

Uma área da TI que tem crescido bastante é a \textit{Business Intelligence} (BI), que reúne uma série de conceitos que podem ser aplicados em empresas, de qualquer tamanho e área, com o objetivo de dar suporte à tomada de decisão, trazendo dados e gerando informação, que serve de base para as escolhas das estratégias de um determinado negócio. \cite{teste}

Esses conceitos devem ser usados para desenvolver um painel para o Centro de Inteligência, que monitora os processos que entram na JFRN, a fim de evitar a multiplicação de demandas repetitivas.

%License''\footnote{\url{http://www.latex-project.org/lppl.txt}}.


% ---
% Capitulo com exemplos de comandos inseridos de arquivo externo 
% ---
%\include{abntex2-modelo-include-comandos}
% ---
\chapter{Conceitos de Business Intelligence}\label{cap_trabalho_academico}

% https://www.researchgate.net/publication/273861123_Why_Business_Intelligence_Significance_of_Business_Intelligence_Tools_and_Integrating_BI_Governance_with_Corporate_Governance

Antes de abordar o problema e a solução em si, é importante que seja definido o que é \textit{Business Intelligence} (BI), os conceitos e aplicações, e também é importante explicar o que é Visualização de Dados. 

O conceito de BI ainda não é bem definido na literatura, mas o que muitos autores e pesquisadores concordam é que ele engloba ferramentas, normalmente de tecnologia da informação, e estratégias que servem para analisar os dados de empresas, gerando informação sobre as organizações. O BI também fornece visões do passado e presente dessas empresas e também podem mostrar dados preditivos. Algumas funções comuns envolvem o processamento analítico online, análise, mineração de dados, gerenciamento de performance do negócio, entre outros. 

Essas técnicas podem processar grandes quantidades de dados estruturados, e algumas vezes não estruturados, que podem ajudar a identificar novas oportunidades de negócios ou implementar novas estratégias, para citar alguns. 
\chapter{Visualização de dados}\label{cap_trabalho_academico}

\chapter{Desenvolvimento do painel}\label{cap_trabalho_academico}

Ao longo do tempo o painel sofreu diversas mudanças. Essas mudanças foram incrementais e uma das principais fontes de exemplos e usos das ferramentas do Dash foi a plataforma Medium, que apresenta vários artigos exemplificando formas de se usar o Dash e como usar melhor os recursos da biblioteca.

Um desses artigos do Medium foi muito importante para a definição de uma estrutura base de desenvolvimento do painel, o texto de Ishan Mehta \cite{medium1} apresenta uma proposta de estrutura que pode ser replicada e melhorada em trabalhos futuros, e a partir dessa estrutura o painel foi montado e desenvolvido, com novas visualizações e diferentes análises.





% Conclusão
\chapter{Conclusão}
Por fim, fica claro que existe uma dificuldade em desenvolver painéis completamente \textit{in-house}, porém as facilidades de usar depois que uma estrutura básica está montada, e a distribuição simplificada tornam o Dash, e o Python, uma ótima alternativa às ferramentas pagas. Com um time não muito grande de TI, com profissionais de \textit{front-end}, análise de dados e infraestrutura é possível atender demandas de diferentes Varas, criando painéis que atendam pedidos locais de visualização de dados, de forma rápida e gratuita.