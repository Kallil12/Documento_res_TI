\chapter{Desenvolvimento do painel}\label{cap_trabalho_academico}

Ao longo do tempo o painel sofreu diversas mudanças. Essas mudanças foram incrementais e uma das principais fontes de exemplos e usos das ferramentas do Dash foi a plataforma Medium, que apresenta vários artigos exemplificando formas de se usar o Dash e como usar melhor os recursos da biblioteca.

Um desses artigos do Medium foi muito importante para a definição de uma estrutura base de desenvolvimento do painel, o texto de Ishan Mehta \cite{medium1} apresenta uma proposta de estrutura que pode ser replicada e melhorada em trabalhos futuros, e a partir dessa estrutura o painel foi montado e desenvolvido, com novas visualizações e diferentes análises.

