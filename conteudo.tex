% ----------------------------------------------------------
% Introdução (exemplo de capítulo sem numeração, mas presente no Sumário)
% ----------------------------------------------------------
\chapter{Introdução}
% ----------------------------------------------------------

Este documento e seu código-fonte são exemplos de referência de uso da classe
\textsf{abntex2} e do pacote \textsf{abntex2cite}. O documento 
exemplifica a elaboração de trabalho acadêmico (tese, dissertação e outros do
gênero) produzido conforme a ABNT NBR 14724:2011 \emph{Informação e documentação
- Trabalhos acadêmicos - Apresentação}.

A expressão ``Modelo Canônico'' é utilizada para indicar que \abnTeX\ não é
modelo específico de nenhuma universidade ou instituição, mas que implementa tão
somente os requisitos das normas da ABNT. Uma lista completa das normas
observadas pelo \abnTeX\ é apresentada em \citeonline{abntex2classe}.

Sinta-se convidado a participar do projeto \abnTeX! Acesse o site do projeto em
\url{http://www.abntex.net.br/}. Também fique livre para conhecer,
estudar, alterar e redistribuir o trabalho do \abnTeX, desde que os arquivos
modificados tenham seus nomes alterados e que os créditos sejam dados aos
autores originais, nos termos da ``The \LaTeX\ Project Public

%License''\footnote{\url{http://www.latex-project.org/lppl.txt}}.


% ---
% Capitulo com exemplos de comandos inseridos de arquivo externo 
% ---
%\include{abntex2-modelo-include-comandos}
% ---
\chapter{Conceitos de Business Intelligence}\label{cap_trabalho_academico}

% https://www.researchgate.net/publication/273861123_Why_Business_Intelligence_Significance_of_Business_Intelligence_Tools_and_Integrating_BI_Governance_with_Corporate_Governance

Antes de abordar o problema e a solução em si, é importante que seja definido o que é \textit{Business Intelligence} (BI), os conceitos e aplicações, e também é importante explicar o que é Visualização de Dados. 

O conceito de BI ainda não é bem definido na literatura, mas o que muitos autores e pesquisadores concordam é que ele engloba ferramentas, normalmente de tecnologia da informação, e estratégias que servem para analisar os dados de empresas, gerando informação sobre as organizações. O BI também fornece visões do passado e presente dessas empresas e também podem mostrar dados preditivos. Algumas funções comuns envolvem o processamento analítico online, análise, mineração de dados, gerenciamento de performance do negócio, entre outros. 

Essas técnicas podem processar grandes quantidades de dados estruturados, e algumas vezes não estruturados, que podem ajudar a identificar novas oportunidades de negócios ou implementar novas estratégias, para citar alguns. 
\chapter{Visualização de dados}\label{cap_trabalho_academico}

\chapter{Desenvolvimento do painel}\label{cap_trabalho_academico}

Ao longo do tempo o painel sofreu diversas mudanças. Essas mudanças foram incrementais e uma das principais fontes de exemplos e usos das ferramentas do Dash foi a plataforma Medium, que apresenta vários artigos exemplificando formas de se usar o Dash e como usar melhor os recursos da biblioteca.

Um desses artigos do Medium foi muito importante para a definição de uma estrutura base de desenvolvimento do painel, o texto de Ishan Mehta \cite{medium1} apresenta uma proposta de estrutura que pode ser replicada e melhorada em trabalhos futuros, e a partir dessa estrutura o painel foi montado e desenvolvido, com novas visualizações e diferentes análises.



\chapter{Conteúdos específicos do modelo de trabalho acadêmico}\label{cap_trabalho_academico}


% Conclusão
\chapter{Conclusão}
\lipsum[31-33]