\chapter{Conceitos de Business Intelligence}\label{cap_trabalho_academico}

% https://www.researchgate.net/publication/273861123_Why_Business_Intelligence_Significance_of_Business_Intelligence_Tools_and_Integrating_BI_Governance_with_Corporate_Governance

De forma geral o termo \textit{Business Intelligence} (BI) ainda não é bem definido na literatura, mas alguns dos principais estudiosos da área, como Solomon Negash, apresentam o BI como um sistema que combina dados operacionais com ferramentas analíticas para apresentar informações para os gerentes do negócio. O objetivo disso é melhorar a qualidade das decisões do processo, tornando-as melhor embasadas, então os métodos do BI podem ser usados para melhor entender o estado da empresa, organização ou órgão onde estiver sendo aplicado, resultando em melhores decisões. Apesar disso, ainda existe bastante empresa que emprega o BI num viés muito mais ligado à gestão, sem exigir muito conhecimento da parte de tecnologia da informação, e usando conceitos sobre mercado, marketing, administração da produção etc. 

Nesse trabalho serão usados os conceitos que se aproximam mais ao que Negash apresenta como BI, e dentro do que ele define como sendo importante na Inteligência de Negócios podemos citar algumas áreas do conhecimento como:

\begin{itemize}
	\item \textit{Data Warehouse}
	\item Visualização
	\item Mineração de dados
	\item \textit{Online Analytic Processing} (OLAP)
	\item Gerenciamento do conhecimento
\end{itemize}

