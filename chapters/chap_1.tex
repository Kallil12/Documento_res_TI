\chapter{Conceitos de Business Intelligence}\label{cap_trabalho_academico}

% https://www.researchgate.net/publication/273861123_Why_Business_Intelligence_Significance_of_Business_Intelligence_Tools_and_Integrating_BI_Governance_with_Corporate_Governance

Antes de abordar o problema e a solução em si, é importante que seja definido o que é \textit{Business Intelligence} (BI), os conceitos e aplicações, e também é importante explicar o que é Visualização de Dados. 

O conceito de BI ainda não é bem definido na literatura, mas o que muitos autores e pesquisadores concordam é que ele engloba ferramentas, normalmente de tecnologia da informação, e estratégias que servem para analisar os dados de empresas, gerando informação sobre as organizações. O BI também fornece visões do passado e presente dessas empresas e também podem mostrar dados preditivos. Algumas funções comuns envolvem o processamento analítico online, análise, mineração de dados, gerenciamento de performance do negócio, entre outros. 

Essas técnicas podem processar grandes quantidades de dados estruturados, e algumas vezes não estruturados, que podem ajudar a identificar novas oportunidades de negócios ou implementar novas estratégias, para citar alguns. 